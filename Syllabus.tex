% Options for packages loaded elsewhere
\PassOptionsToPackage{unicode}{hyperref}
\PassOptionsToPackage{hyphens}{url}
\PassOptionsToPackage{dvipsnames,svgnames,x11names}{xcolor}
%
\documentclass[
]{article}
\usepackage{amsmath,amssymb}
\usepackage{lmodern}
\usepackage{iftex}
\ifPDFTeX
  \usepackage[T1]{fontenc}
  \usepackage[utf8]{inputenc}
  \usepackage{textcomp} % provide euro and other symbols
\else % if luatex or xetex
  \usepackage{unicode-math}
  \defaultfontfeatures{Scale=MatchLowercase}
  \defaultfontfeatures[\rmfamily]{Ligatures=TeX,Scale=1}
\fi
% Use upquote if available, for straight quotes in verbatim environments
\IfFileExists{upquote.sty}{\usepackage{upquote}}{}
\IfFileExists{microtype.sty}{% use microtype if available
  \usepackage[]{microtype}
  \UseMicrotypeSet[protrusion]{basicmath} % disable protrusion for tt fonts
}{}
\makeatletter
\@ifundefined{KOMAClassName}{% if non-KOMA class
  \IfFileExists{parskip.sty}{%
    \usepackage{parskip}
  }{% else
    \setlength{\parindent}{0pt}
    \setlength{\parskip}{6pt plus 2pt minus 1pt}}
}{% if KOMA class
  \KOMAoptions{parskip=half}}
\makeatother
\usepackage{xcolor}
\usepackage[margin=1in]{geometry}
\usepackage{graphicx}
\makeatletter
\def\maxwidth{\ifdim\Gin@nat@width>\linewidth\linewidth\else\Gin@nat@width\fi}
\def\maxheight{\ifdim\Gin@nat@height>\textheight\textheight\else\Gin@nat@height\fi}
\makeatother
% Scale images if necessary, so that they will not overflow the page
% margins by default, and it is still possible to overwrite the defaults
% using explicit options in \includegraphics[width, height, ...]{}
\setkeys{Gin}{width=\maxwidth,height=\maxheight,keepaspectratio}
% Set default figure placement to htbp
\makeatletter
\def\fps@figure{htbp}
\makeatother
\setlength{\emergencystretch}{3em} % prevent overfull lines
\providecommand{\tightlist}{%
  \setlength{\itemsep}{0pt}\setlength{\parskip}{0pt}}
\setcounter{secnumdepth}{-\maxdimen} % remove section numbering
\usepackage{booktabs}
\usepackage{xcolor}
\ifLuaTeX
  \usepackage{selnolig}  % disable illegal ligatures
\fi
\usepackage[]{natbib}
\bibliographystyle{plainnat}
\IfFileExists{bookmark.sty}{\usepackage{bookmark}}{\usepackage{hyperref}}
\IfFileExists{xurl.sty}{\usepackage{xurl}}{} % add URL line breaks if available
\urlstyle{same} % disable monospaced font for URLs
\hypersetup{
  pdftitle={Text as data - syllabus},
  pdfauthor={Max Callaghan},
  colorlinks=true,
  linkcolor={Maroon},
  filecolor={Maroon},
  citecolor={Blue},
  urlcolor={blue},
  pdfcreator={LaTeX via pandoc}}

\title{Text as data - syllabus}
\author{Max Callaghan}
\date{2022-09-08}

\begin{document}
\maketitle

\hypertarget{course-contents-and-learning-objectives}{%
\section{Course contents and learning
objectives}\label{course-contents-and-learning-objectives}}

\hypertarget{course-contents}{%
\subsection{Course contents:}\label{course-contents}}

There is an abundance of unstructured data around us. Working with text
is key, not just to measure discussions and opinions on social media or
in product reviews, but also to gain insights into concepts important to
the study of politics such as ideological positions or policy
sentiments. The vast amount of textual data that one frequently needs to
process, and the messy form that it often comes in, poses special
challenges to researchers. This course introduces students to
computational tools and methods that enable them to treat text as data.
We will touch upon core theoretical concepts, but the main goal is to
give you hands-on experience in using R to collect, load, prepare and
analyze text data.

\hypertarget{main-learning-objectives}{%
\subsection{Main learning objectives}\label{main-learning-objectives}}

Students will gain hands-on experience working with different types of
text data in R. This includes obtaining, managing, and wrangling data,
as well as applying different models for analysing text

\hypertarget{target-group}{%
\subsection{Target group}\label{target-group}}

The course is aimed at students with an interest in programming, who
wish to gain experience analysing large collections of texts.

\hypertarget{prerequisites}{%
\subsection{Prerequisites}\label{prerequisites}}

Basic knowledge of R is required

\hypertarget{diversity-statement}{%
\subsection{Diversity Statement}\label{diversity-statement}}

Understanding and respect for all cultures and ethnicities is central to
the teaching at Hertie. Being mindful of diversity is an important issue
for policy professionals in the planning, implementation, and evaluation
of programmes designed for specific groups, populations, or communities.
Diversity and cultural awareness will be integrated in the course
content whenever possible.

\hypertarget{grading-and-assignments}{%
\subsection{Grading and Assignments}\label{grading-and-assignments}}

\begin{tabular}{p{0.28\linewidth}p{0.25\linewidth}p{0.2\linewidth}p{0.1\linewidth}}

\toprule[1pt]

\textbf{Assignment 1: Programming exercise} & Deadline: 13 October & Submit via Moodle & 30\% \\
\midrule[0.5pt]
\textbf{Assignment 2: data analysis exercise} & Deadline:17  November & Submit via Moodle & 30\% \\

\midrule[0.5pt]

\textbf{Assignment 3: Oral presentation of your own research project} & Deadline 1 December & Presentation in class & 40\% \\
\bottomrule[0.5pt]


\end{tabular}

\hypertarget{assignment-details}{%
\subsection{Assignment Details}\label{assignment-details}}

\hypertarget{assignment-1}{%
\subsubsection{Assignment 1}\label{assignment-1}}

The first assignment will be a programming exercise, where you will
receive a dataset and clear step-by-step instructions to import and
manage text, and produce a simple analysis. Using what we have learnt in
class, you will be required to write a script to carry out the
instructions.

\hypertarget{assignment-2}{%
\subsubsection{Assignment 2}\label{assignment-2}}

The second assignment will involve the construction of a topic model,
including a visualization and brief discussion of the results. Your
grade for this assignment will be based on how you approach this task
and if you follow the steps discussed in class.

\hypertarget{assignment-3}{%
\subsubsection{Assignment 3}\label{assignment-3}}

Your final assignment is the presentation of a group research project.
You are free to choose your subject and methods, as long as the project
involves the analysis of text and that the methods are covered in this
course. Grading will be determined by the quality of the presentation,
and the degree to which you manage to apply the skills you have learned
during the course.

\hypertarget{late-submission-of-assignments}{%
\subsection{Late submission of
assignments}\label{late-submission-of-assignments}}

For each day the assignment is turned in late, the grade will be reduced
by 10 percentage points.

\hypertarget{attendance}{%
\subsection{Attendance}\label{attendance}}

Students are expected to be present and prepared for every class
session. Active participation during lectures and seminar discussions is
essential. If unavoidable circumstances arise which prevent attendance
or preparation, the instructor should be advised by email with as much
advance notice as possible. Please note that students cannot miss more
than two out of 12 course sessions. For further information please
consult the
\href{https://moodle.hertie-school.org/mod/book/view.php?id=47912}{Examination
Rules} §10.

\hypertarget{academic-integrity}{%
\subsection{Academic Integrity}\label{academic-integrity}}

The Hertie School is committed to the standards of good academic and
ethical conduct. Any violation of these standards shall be subject to
disciplinary action. Plagiarism, deceitful actions as well as
free-riding in group work are not tolerated. See
\href{https://moodle.hertie-school.org/mod/book/view.php?id=47912}{Examination
Rules} §16 and the
\href{https://moodle.hertie-school.org/mod/page/view.php?id=66059}{Hertie
Plagiarism Policy}.

\hypertarget{compensation-for-disadvantages}{%
\subsection{Compensation for
Disadvantages}\label{compensation-for-disadvantages}}

If a student furnishes evidence that he or she is not able to take an
examination as required in whole or in part due to disability or
permanent illness, the Examination Committee may upon written request
approve learning accommodation(s). In this respect, the submission of
adequate certificates may be required. See
\href{https://moodle.hertie-school.org/mod/book/view.php?id=47912}{Examination
Rules} §14.

\hypertarget{extenuating-circumstances}{%
\subsection{Extenuating circumstances}\label{extenuating-circumstances}}

An extension can be granted due to extenuating circumstances (i.e., for
reasons like illness, personal loss or hardship, or caring duties). In
such cases, please contact the course instructors and the Examination
Office in advance of the deadline.

\hypertarget{general-readings}{%
\section{General Readings}\label{general-readings}}

Silge, Julia, and David Robinson. 2017. Text mining with R: A tidy
approach. O'Reilly Media, Inc.~\url{https://www.tidytextmining.com/}

Wickham, H. and G. Grolemund. 2017. R for Data Science: Import, Tidy,
Transform, Visualize, and Model Data. O'Reilly.
\url{https://r4ds.had.co.nz/}

\hypertarget{session-overview}{%
\section{Session Overview}\label{session-overview}}

\hypertarget{session-1-text-as-data}{%
\subsection{Session 1: Text as data}\label{session-1-text-as-data}}

The first session provides a general introduction to the subject, and
outlines how we will proceed with the course.

\textbf{Readings}: Grimmer and Stewart (2013). `Text as Data: The
Promise and Pitfalls of Automatic Content Analysis Methods for Political
Documents'. Political Analysis.
\href{https://www.cambridge.org/core/journals/political-analysis/article/text-as-data-the-promise-and-pitfalls-of-automatic-content-analysis-methods-for-political-texts/F7AAC8B2909441603FEB25C156448F20}{link}

\hypertarget{session-2-preprocessing-and-turning-texts-into-features}{%
\subsection{Session 2: Preprocessing, and turning texts into
features}\label{session-2-preprocessing-and-turning-texts-into-features}}

This week, we will learn how we turn texts into features, and consider
the choices we can make in doing so

\textbf{Readings}: Watanabe, Kohei and Stefan Müller. 2022. ``Quanteda
Tutorials''. Chapter 3.
\url{https://tutorials.quanteda.io/basic-operations/}.

\hypertarget{session-3-acquiring-importing-and-reading-texts}{%
\subsection{Session 3: Acquiring, importing, and reading
texts}\label{session-3-acquiring-importing-and-reading-texts}}

In the third session we will cover how to acquire, and import text data,
and we will demonstrate how to retrieve texts from APIs

\textbf{Readings:}

Bail, Chris. 2019. Text as Data: Application Programming Interfaces in
R.
\url{https://cbail.github.io/textasdata/apis/rmarkdown/Application_Programming_interfaces.html}

\hypertarget{session-4-manipulating-strings}{%
\subsection{Session 4: Manipulating
strings}\label{session-4-manipulating-strings}}

This session will introduce you to regex expressions, and explain how
the stringr package can be used to manipulate strings.

\textbf{Readings:}

Wickham, H. and G. Grolemund. 2017. R for Data Science: Import, Tidy,
Transform, Visualize, and Model Data. O'Reilly. Chapter 14

Wickham, Hadley. 2010. ``stringr: modern, consistent string
processing''. The R Journal 2 (2): 38-40.

\hypertarget{session-5-visualising-text-features-with-ggplot}{%
\subsection{Session 5: Visualising text features with
ggplot}\label{session-5-visualising-text-features-with-ggplot}}

In this session, we will explore how to visualise the features we have
generated from texts, and make graphs that show how corpora differ, or
how they have changed over time.

\textbf{Readings:}

Silge, Julia, and David Robinson. 2017. Text mining with R: A tidy
approach. O'Reilly Media, Inc.~\url{https://www.tidytextmining.com/}.
Chapter 3.

\colorbox{yellow}{Assignment 1 set}

\hypertarget{session-6-word-embeddings}{%
\subsection{Session 6: Word
embeddings}\label{session-6-word-embeddings}}

In this session, we will learn about how word embeddings can offer
richer representations of the content of texts

\textbf{Readings:}

Hvitfeldt, Emil, and Silge, Julia. 2019. Supervised Machine Learning for
Text Analysis in R. Chapter 5 - Word Embeddings.
\url{https://smltar.com/embeddings.html}

\colorbox{pink}{Assignment 1 due}

\hypertarget{session-7-dimensionality-reduction}{%
\subsection{Session 7: Dimensionality
reduction}\label{session-7-dimensionality-reduction}}

This session will demonstrate ways to represent multidimensional data in
a 2-dimensional space.

\textbf{Readings:}

Conlen, Matthew, and Hohman, Fred. 2018. The Beginner's Guide to
Dimensionality Reduction.
\url{https://dimensionality-reduction-293e465c2a3443e8941b016d.vercel.app/}

\hypertarget{session-8-topic-modelling}{%
\subsection{Session 8: Topic
modelling}\label{session-8-topic-modelling}}

This session will introduce topic models, and show how these can be run
and visualised in R.

\textbf{Readings:}

Blei, David M. 2012. Probabilistic Topic Models. Communications of the
ACM. 55 (4) pp 77--84 \url{https://doi.org/10.1145/2133806.2133826}
\url{http://www.cs.columbia.edu/~blei/papers/Blei2012.pdf}

\colorbox{yellow}{Assignment 2 set}

\hypertarget{session-9-sentiment-analysis}{%
\subsection{Session 9: Sentiment
analysis}\label{session-9-sentiment-analysis}}

This session will demonstrate how we can analyse the sentiment of texts
using R.

\textbf{Readings:}

Silge, Julia, and David Robinson. 2017. Text mining with R: A tidy
approach. O'Reilly Media, Inc.~\url{https://www.tidytextmining.com/}.
Chapter 2.

\hypertarget{session-10-supervised-learning}{%
\subsection{Session 10: Supervised
learning}\label{session-10-supervised-learning}}

This session will demonstrate how we can use R to train machine learning
models to reproduce a set of labels applied to texts

\textbf{Readings:}

Hvitfeldt, Emil, and Silge, Julia. 2022. Supervised Machine Learning for
Text Analysis in R. CRC Press. \url{https://smltar.com/}. Chapter 7.

\colorbox{pink}{Assignment 2 due}

\hypertarget{session-11-spacy-and-transformers}{%
\subsection{Session 11: Spacy and
Transformers}\label{session-11-spacy-and-transformers}}

This session will explore some of the latest developments in NLP, and
show some of the capabilities of Spacy and Transformers.

\textbf{Readings:}

Devlin, Jacob, et al.~2019. ``Bert: Pre-training of deep bidirectional
transformers for language understanding.'' Proceedings of NAACL-HLT:
4171--4186.

Alammar, Jay. The Illustrated Transformer:
\url{https://jalammar.github.io/illustrated-transformer/}

\hypertarget{session-12-wrapup-and-final-presentations}{%
\subsection{Session 12: Wrapup and final
presentations}\label{session-12-wrapup-and-final-presentations}}

In the final session, we will hear the presentations of the group
projects you have conducted, and do a final wrap-up of what we have
learned.

\colorbox{pink}{Assignment 3 due}

  \bibliography{../presentation-resources/MyLibrary.bib}

\end{document}
